% --------------------------------------------------------------------------- %
% Author:          Joey Dumont                <joey.dumont@gmail.com>         %
% Date created:    Aug. 14th, 2017                                            %
% Description:     Prepared for ExHILP 2017.                                  %
%                  TikZ code to draw the evolution of a VSF parabola.         %
% License:         CC0                                                        %
%                  <https://creativecommons.org/publicdomain/zero/1.0>        %
% --------------------------------------------------------------------------- %

\documentclass[tikz]{standalone}
\usepackage[charter]{mathdesign}
\usepackage{amsmath}
\usepackage{xifthen}
\usepackage{fontspec}
\setmainfont{Oswald}


\definecolor{inrs_blue}{RGB}{20, 107, 182}


% -- Change this value to change the shape of the parabola.
\newcommand{\focal}{0.5}
\newcommand{\rmax}{1}

\newcommand{\parabola}[1]{\pgfmathsetmacro{\myvalue}{#1*#1/(4*\focal)-\focal}}
\newcommand{\iparabola}[1]{\pgfmathsetmacro{\myivalue}{sqrt(4*\focal*(#1+\focal))}}
\parabola{\rmax}
\pgfmathsetmacro{\zmax}{\myvalue}
\pgfmathsetmacro{\rmin}{0}
\pgfmathsetmacro{\zmaxplot}{1.52}

\begin{document}

\begin{tikzpicture}[scale=3]

  \parabola{0}
  \newcommand{\offset}{0.1}
  \newcommand{\laserLength}{2}
  \draw[domain=-\focal:-\offset,samples=1000,line width=0.0mm,white,fill=inrs_blue,opacity=0.15,draw=none] plot (\x,{sqrt(4*\focal*(\x+\focal))}) -- ++(\laserLength,0) -- ++(0,2*\myvalue+\offset) -- ++(-\laserLength,0) -- cycle;
  %\draw[domain=-\focal:-\offset,samples=1000,line width=0.0mm,white,fill=inrs_blue,opacity=0.15,draw=none] plot (\x,{-sqrt(4*\focal*(\x+\focal))}) -- ++(\laserLength,0) -- ++(0,-2*\myvalue-0.53) -- ++(-\laserLength,0) -- cycle;

  %\draw[domain=-\focal:-\offset,samples=1000,line width=0.0mm,white,fill=inrs_blue,opacity=0.15,draw=none] plot (\x,{-sqrt(4*\focal*(\x+\focal))}) -- (0,0) -- (-\focal,0) -- cycle;
  %\draw[domain=-\focal:-\offset,samples=1000,line width=0.0mm,white,fill=inrs_blue,opacity=0.15,draw=none] plot (\x,{sqrt(4*\focal*(\x+\focal))}) -- (0,0) -- (-\focal,0) -- cycle;


  % -- This draws the parabolic mirror.
  \draw[domain=-\focal:\zmax,samples=1000,ultra thick] plot (\x,{sqrt(4*\focal*(\x+\focal))});
  \draw[domain=-\focal:\zmax,samples=1000,ultra thick] plot (\x,{-sqrt(4*\focal*(\x+\focal))});

  % -- This draws the position of the focal plane.
  \draw[dashed] (0,-\rmax) -- (0,\rmax);

  \foreach \x in {-0.80,-0.70,...,0.80}
  {
    \parabola{\x}\draw (\zmaxplot,\x) -- (\myvalue,\x) -- (0,0);
  }

  \draw[gray] (-1.2,0) -- (\zmaxplot, 0) node [at start, above right] {Optical axis};

  \parabola{\rmax}
  \clip (-0.8,\rmax) rectangle (\zmaxplot,-\rmax);

  % -- Draw the lengths.
  %\draw[<->,>=stealth] (-\focal,0) -- (0,0) node[midway,fill=white] {$f$};
  %\draw[<->,>=stealth] (\zmax,-\rmax) -- (\zmax,\rmax) node[midway,fill=white] {$r_\text{max}$};

\end{tikzpicture}

\end{document}
