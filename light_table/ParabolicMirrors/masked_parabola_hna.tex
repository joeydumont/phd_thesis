% --------------------------------------------------------------------------- %
% Author:          Joey Dumont                <joey.dumont@gmail.com>         %
% Date created:    Jun 19th, 2015                                             %
% Description:     Prepared for my doctoral thesis.                           %
%                  TikZ code to show the masked parabola.                     %
% License:         CC0                                                        %
%                  <https://creativecommons.org/publicdomain/zero/1.0>        %
% --------------------------------------------------------------------------- %

\documentclass[tikz]{standalone}
\usepackage[charter]{mathdesign}
\usepackage{amsmath}
\usepackage{xifthen}
\usepackage{fontspec}
\setmainfont{Oswald}

\usetikzlibrary{calc,intersections,patterns}

% -- Parameters of the parabola.
\newcommand{\focal}{0.5}
\newcommand{\rmax}{1}

% -- Functions that compuzte z(r), or r(z).
\newcommand{\parabola}[1]{\pgfmathsetmacro{\myvalue}{#1*#1/(4*\focal)-\focal}}
\newcommand{\iparabola}[1]{\pgfmathsetmacro{\myivalue}{sqrt(4*\focal*(#1+\focal))}}

% -- How much of the parabola we plot.
\parabola{\rmax}
\pgfmathsetmacro{\zmax}{\myvalue}
\pgfmathsetmacro{\rmin}{0}
\pgfmathsetmacro{\zmaxplot}{1.52}

% -- Extract angle between two coordinates. The first
% -- argument is the output value.
\newcommand{\pgfextractangle}[3]{%
    \pgfmathanglebetweenpoints{\pgfpointanchor{#2}{center}}
                              {\pgfpointanchor{#3}{center}}
    \global\let#1\pgfmathresult
}

\begin{document}

\begin{tikzpicture}[scale=3]

  % -- This draws the parabola.
  \draw[domain=-\focal:\zmax,samples=1000,ultra thick,smooth,name path=parabola1] plot (\x,{sqrt(4*\focal*(\x+\focal))});
  \draw[domain=-\focal:\zmax,samples=1000,ultra thick,smooth,name path=parabola2] plot (\x,{-sqrt(4*\focal*(\x+\focal))});

  % -- This draws the position of the focal plane.
  \draw[dashed] (0,-\rmax) -- (0,\rmax);

  % -- Denotes the position of the optical axis.
  \draw[gray] (-1.2,0) -- (\zmaxplot, 0) node [at start, above right] {Optical axis};

  % -- This draws the light rays on parabola1.
  \foreach \x in {-0.80,-0.70,...,0.60,0.80}
  {
    \parabola{\x}\draw (\zmaxplot,\x) -- (\myvalue,\x) -- (0,0);
  }


  % -- Prepares the endpoints for the shadow.
  \pgfmathsetmacro{\rmaxsha}{0.75}
  \path [name path=line2](1.5, \rmaxsha)--(-0.5,\rmaxsha);
  \path [name path=line2a, name intersections={of=parabola1 and line2}] (1.5, \rmaxsha) -- (intersection-1);
  \coordinate[] (A2)  at (intersection-1);

  \pgfmathsetmacro{\rminsha}{0.55}
  \path [name path=line3](1.5, \rminsha)--(-0.5,\rminsha);
  \path [name path=line3a,name intersections={of=parabola1 and line3}] (1.5, \rminsha) -- (intersection-1);
  \coordinate[] (A3)  at (intersection-1);

  % -- Draws the shadow from the mask to the parabola.
  \parabola{\rminsha}\pgfmathsetmacro{\zminsha}{\myvalue}
  \parabola{\rmaxsha}\pgfmathsetmacro{\zmaxsha}{\myvalue}
  \draw[domain=\zminsha:\zmaxsha,samples=1000,line width=0.0mm,white,fill=gray,opacity=0.2,draw=none] plot (\x,{sqrt(4*\focal*(\x+\focal))}) -- (1.5, \rmaxsha) -- (1.5, \rminsha) -- (A3) -- cycle;

  % -- Draw the mask.
  \draw[fill=black!50] (1.30, \rminsha) rectangle (1.35, \rmaxsha);

  % -- Draw the lines ending on the mask.
  \foreach \x in {0.6,0.7}
  {
    \draw (\zmaxplot,\x) -- (1.35, \x);
  }

  % -- Drwas the shadow from the parabola to the focal spot.
  \draw[domain=\zminsha:\zmaxsha,samples=1000,line width=0.0mm,white,fill=gray,opacity=0.2,draw=none] plot (\x,{sqrt(4*\focal*(\x+\focal))}) -- (0.0,0.0) -- (A3) -- cycle;

  % -- Draws the shadow from the focal spot to the detector.
  \coordinate[] (S1)  at ($(0,0)!-1.2cm!(A2)$);
  \coordinate[] (S2)  at ($(0,0)!-1.2cm!(A3)$);
  \draw[samples=1000,line width=0.0mm,white,fill=gray,opacity=0.2,draw=none] (0.0,0.0) -- (S1) -- (S2) -- cycle;

\end{tikzpicture}

\end{document}
