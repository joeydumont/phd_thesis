% --------------------------------------------------------------------------- %
% Author:          Joey Dumont                <joey.dumont@gmail.com>         %
% Date created:    Mar. 8th, 2017                                             %
% Description:     Ph. D. thesis file.                                        %
% License:         CC0                                                        %
%                  <https://creativecommons.org/publicdomain/zero/1.0>        %
% --------------------------------------------------------------------------- %

% --------------------------------------------------------------------------- %
% --                               Preamble                                -- %
% --------------------------------------------------------------------------- %

% ----------------------------------------------------------------- %
% --                       Document Class                        -- %
% ----------------------------------------------------------------- %

\documentclass[11pt,SymmetricalJury]{inrsthesis}

% ----------------------------------------------------------------- %
% --                          Packages                           -- %
% ----------------------------------------------------------------- %

\usepackage[unicode=true,
      pdfauthor={Joey Dumont},
      pdftitle={},
      bookmarks=true,
      bookmarksnumbered=true,
      bookmarksopen=true,
      bookmarksopenlevel=3,
      breaklinks=false,
      pdfborder={0 0 0},
      backref=false,
      colorlinks=true,
      linktoc=page,
      linkcolor=red,
      citecolor=blue,
      urlcolor=blue]
 {hyperref}

\usepackage[showframe,pass]{geometry} % Shows the frame properly when using memoir.
\usepackage[theorems=false]{latex-tools/vphys}        % Common LaTeX tools that I use.
\usepackage[charter]{mathdesign}
\usepackage{todonotes}

% ----------------------------------------------------------------- %
% --                        Customization                        -- %
% ----------------------------------------------------------------- %
\newenvironment{chaptersummary}{%
  \begin{quotation}
  \SingleSpacing
  \setlength{\parskip}{\baselineskip}}{%
  \end{quotation}}

\allowdisplaybreaks

% ----------------------------------------------------------------- %
% --                       Title Page Info                       -- %
% ----------------------------------------------------------------- %
\title{Strong-field quantum electrodynamics \\ in tightly focused fields}
%\subtitle{Towards a Realitic Modelling of High-Power \\Laser Systems in the Quantum Theory}
\author{Joey Dumont}
\year{2018}
\program{Sciences de l'énergie et des matériaux}
\centreINRS{Centre Énergie Matériaux et Télécommunications}
\jury{
  \juryitem
    {Président du jury et \\ examinateur interne}
    {Nom du professeur \\ Institution}
  \\
  \juryitem
    {Examinateur externe}
    {Nom du professeur \\ Faculté ou département \\ Institution}
  \\
  \juryitem
    {Examinateur interne}
    {Nom du professeur \\ Institution}
  \\
  \juryitem
    {Directeur de recherche}
    {Jean-Claude Kieffer \\ INRS-ÉMT}
  \\
  \juryitem
    {Codirecteur de recherche}
    {Steve MacLean \\ INRS-ÉMT \\ University of Waterloo}
}
% --------------------------------------------------------------------------- %
% --                               Document                                -- %
% --------------------------------------------------------------------------- %

\begin{document}

\frontmatter

\maketitle

\chapter{Résumé}

Mots-clés:

\chapter{Abstract}

Keyword:

\chapter{Sommaire récapitulatif}
\cleardoublepage

\tableofcontents
\cleardoublepage

\listoftables
\cleardoublepage

\listoffigures
\cleardoublepage

% \dedication{To my wife and son.}
% \cleardoublepage

\epigraph{More important, the formulation in terms of a variational principle is the route
          that is generally followed when we try to describe apparently nonmechanical systems
          in the mathematical clothes of classical mechanics, as in the theory of fields.}{Goldstein \textit{et al}, Classical Mechanics, 3rd edition.}
\cleardoublepage

\epigraph{Whatever you do, do the best you can. The film lives forever!}{Jackie Chan}

\begin{flushright}
\begin{minipage}{0.35\textwidth}
The author claims that this statement not only applies for films with badass
action sequences, but for every artistic, cultural or scientific endeavour.
\end{minipage}
\end{flushright}

\cleardoublepage
\mainmatter

\chapter{Introduction}

Basic plan:
  \begin{itemize}
    \item Generic introduction to SF-QED, experiments by SLAC and the upcoming
          high power-laser facilities.
    \item Motivation to accurately model the tight focusing regime: field inhomogeneities
          play an important role in the detection of SF-QED observables (cite recent work by Di Piazza).
    \item Approach based on two steps: modelling the optical side of things, then the quantum side of things.
    \item Motivation behind the StrattoCalculator (vs Richards-Wolf, vs analytical solutions).
    \item Motivation behind SK (vs effective methods, vs analytical solutions). BONUS: inclusive observables.
    \item Discussion of the use of HPC resources as a driver of future SF-QED experiments?
  \end{itemize}

Stuff to put in appendices:
  \begin{enumerate}
    \item Details of the Stratton-Chu formalism, perhaps some stuff on
          discontinuity ``problem'' \cite{Asvestas1980} and the heuristic argument as to why
          the physical approximation is correct.
    \item Expressions of identity operators in Fock space for photons and fermions.
    \item Proof that disconnected vacuum diagrams vanish identically in the SK formalism.
    \item Minor results related to the \texttt{StrattoCalculator}:
      \begin{itemize}
        \item Gouy phase and its transverse momentum interpretation.
        \item Work on the generation of pseudo-radially polarized beams with mosaics of half-wave plates?
      \end{itemize}
  \end{enumerate}

\chapter{Tool 1: The Stratton-Chu Diffraction Integrals}
\label{chapter:stratton-chu}

\begin{chaptersummary}
This chapter discusses the problem of properly modelling electromagnetic fields
in the tight focusing regime. We first lay the foundation of electromagnetic
scattering and justify the choice of the Stratton-Chu formalism. Our numerical
implementation of this formalism is then discussed at length, and our results
are benchmarked against analytical results and results from the literature.
We briefly discuss the Richards-Wolf formalism, which is similar but not equivalent
to the Stratton-Chu formulation.
\end{chaptersummary}

IDEAS:
  \begin{itemize}
    \item Example of laser chain diagram with regime of validity of paraxial approximation.
    \item HNA parabolic mirror as counterexample in the laser chain, and danger of
            sending the laser back into itself through secondary reflections.
    \item
  \end{itemize}

As mentioned in the introduction of this thesis, focusing electromagnetic fields
down to a volume approximately the size of the wavelength $\lambda$ of the radiation
is a major ingredient in the journey to ever higher intensities. In this regime where
the spatial extension of the field is on the same order of magnitude as its wavelength,
the very common paraxial approximation does not hold. This approximation can be applied
to fields with small diffraction angles $\epsilon=\lambda/\pi w_0$ where $w_0$ is the
transverse width of the beam.

Skeleton:
  \begin{itemize}
      \item Short derivation of the Stratton-Chu equations, with emphasis on the physical optics approximation.
      \item Numerical implementation of the method
        \begin{itemize}
            \item Pros/cons vs more popular methods.
            \item Details of the field calculation (field models, normalization and whatnot).
            \item Details of the MPI implementation.
            \item Phase cancellation in the focal spot.
            \item Appendix: Numerical checks like convergence, solution of Maxwell's equations.
            \item Appendix: Far-field of the SC equations.
            \item Appendix: Analysis of the Gouy phase and its transverse momentum interpretation.
        \end{itemize}
      \item Comparison with Richards-Wolf:
        \begin{itemize}
            \item No ``real'' derivation from SC to RW without implicit approximations.
            \item Problem with phase cancellation?
            \item Differing results in similar conditions.
        \end{itemize}
      \item Analysis of the fields in the focal spot and comparison with
            existing analytical solutions (Salamin, e-dipoles).
      \item Opening to other methods that are less computationnaly intensive;
        \begin{itemize}
            \item Transformation optics;
            \item Propagation equations?
        \end{itemize}
  \end{itemize}


\chapter{Application I: Four-Wave Mixing \textit{in vacuo} and Pair Production}

Skeleton:
\begin{itemize}
    \item Simple derivation of the EH Lagrangian.
    \item How to extract FWM from EH: linearisation + integral solution (WaveMixer)
    \item How to extract PP from EH: Adiabatic formula.
    \item MASTER TABLE, or how different field models affect FWM.
      \begin{itemize}
        \item Introduction of the different field models.
        \item Introduction of the transmission parabola.
        \item Explain the behaviour of yield as a function of 2f/rmax with Lorentz invariants,
              i.e. the optics of the experiment.
      \end{itemize}
    \item Pair production, w/ linear and radial beams.
      \begin{itemize}
          \item Show that both polarizations show the same behaviour as a function of f.
          \item Visualize the 3D fields and explain optimum in terms of the optics (cancellation of the
                magnetic field in both cases).
      \end{itemize}
\end{itemize}

\chapter{Application II: Radiation Reaction With High Energy Electrons}

Skeleton (still a rough idea):
  \begin{itemize}
    \item Quantify RR in different field configurations, checking whether there
          is an advantage to using a transmission parabola in this case.
    \item Study the effects of different field models on the detectability of
          RR effects.
    \item Mention intensity measurement stuff? I've worked a little on this\ldots
  \end{itemize}

\chapter{Tool 2: The Schwinger-Keldysh Formalism and SF-QED Observables}

% On the mathematical nonexistence of path integrals.
%Goldstein, 3rd ed. p. 36
\epigraph{The emperor has no clothes.}{Yours truly}


Skeleton (early stages):
  \begin{itemize}
      \item Motivation of SK: single+double Compton scattering in Furry picture,
            difficulty of going above double/including the effects of arbitrary fields.
      \item Generalities of the SK formalism (definitions and formal results).
      \item Application to SF-QED: computing inclusive observables in
            arbitrary fields (proof-of-concept derivation following Gelis).
      \item Discuss the generality of the method: compatibility with
            Volkov solns, with the Di Piazza method, \textit{and} with numerical
            solutions of the Dirac equation.
      \item Appendix: Disconnected vacuum diagrams vanish in SK.
      \item Appendix: Proper derivation of the Volkov-SK propagators.
  \end{itemize}


\chapter{Application III: Photon Production via Compton Scattering}

Work to be done:
  \begin{itemize}
      \item Formal derivation of the average number of photons with an electron
            in the initial state.
      \item Computation with Volkov propagators.
      \item Computation with Di Piazza solution.
      \item Opening to numerical solutions?
\end{itemize}

\chapter{Application IV: Pair Production via Compton Scattering (Trident/Breit-Wheeler)}

Work to be done:
See previous chapter but with number of pairs through ``secondary'' processes.

\chapter{Conclusion}

%\backmatter

\bibliographystyle{osajnl}
\bibliography{phd_thesis}

\appendix

\chapter{Explicit expressions of the Stratton-Chu equations}

In chapter \ref{chapter:stratton-chu}, we discussed the numerical evaluation
of the Stratton-Chu equations as a way to model electromagnetic fields in the
tightly focused regime. Specifically, we implemented a program that evaluates
\eqref{} for arbitrary mirror geometries, arbitrary incident fields and capable
of modeling broadband pulses. In this appendix, we provide explicit expressions
of the Stratton-Chu equations in cylindrical coordinates and write down a far-field
approximation of those equations.

\section{Explicit expression in cylindrical coordinates}

Our starting point is \eqref{}, which we print here for convenience
  \begin{subequations}
  \label{eq:app.sc-explicit.poa-og}
  \begin{align}
    \label{eq:app.sc-explicit.poa-og.efield}
    \vb{E}'_\text{ref}(\vb{r}',k)  &= 2\iint_S
      \left\{
        ik(\vu{n}\times\vb{B}_\text{inc})g
        +\left(\vu{n}\cdot\vb{E}_\text{inc}\right)\nabla g\right\}dS
            + \frac{2}{ik}\oint_{\partial S}\nabla g
                    \vb{B}\cdot d\vb{\ell},\\
                    %\left[\vu{n}\times(\vu{n}\times\vb{B}_\text{inc})\right]\cdot d\vb{\ell},\\
    \label{eq:app.sc-explicit.poa-og.bfield}
    \vb{B}'_\text{ref}(\vb{r}',k)  &= 2\iint_S (\vu{n}\times\vb{B}_\text{inc})\times\nabla g dS.
  \end{align}
  \end{subequations}
where recall that
  \begin{align}
    g(\vb{r},\vb{r}') &= \frac{e^{ik\left|\vb{r}-\vb{r}'\right|}}{\left|\vb{r}-\vb{r}'\right|}, \\
    \left|\vb{r}-\vb{r}'\right| &= \sqrt{r^2+r'^2-2rr'\cos(\theta-\theta')-(z-z')^2}.
  \end{align}

Before we proceed, we'll show some useful intermediate results that will ease the parsing
of this expression. First, note that we compute the gradient of the Green's
function w.r.t. to the unprimed coordinates such that
  \begin{equation}
    \nabla g = \left(\frac{ik}{u}-\frac{1}{u^2}\right)\left[\left(r-r'\cos(\theta-\theta')\right)\vu{r}+r'\sin(\theta-\theta')\vu{\theta}+(z-z')\vu{z}\right]g = \left(\frac{ik}{u}-\frac{1}{u^2}\right)g\vb{G}
  \end{equation}
Also note that to extract the cylindrical components of the electromagnetic
field vectors, it will be necessary to evaluate the dot product of
\eqref{eq:app.sc-explicit.poa-og} with cylindrical basis vectors evaluated
at $\vb{r}'$ with cylindrical basis vectors evaluated at $\vb{r}$.
Because the cylindrical basis vectors form a non-coordinate basis, these
dot products do not vanish. In fact, we have
  \begin{subequations}
  \begin{align}
    \vu{r}'\cdot\vu{r}            &= (\cos\theta'\vu{x}+\sin\theta'\vu{y})\cdot(\cos\theta\vu{x}+\sin\theta\vu{y})
                                   = \cos\theta'\cos\theta+\sin\theta'\sin\theta
                                   = \cos(\theta-\theta') \\
    \vu{r}'\cdot\vu{\theta}       &= (\cos\theta'\vu{x}+\sin\theta'\vu{y})\cdot(-\sin\theta\vu{x}+\cos\theta\vu{y})
                                   = -\cos\theta'\sin\theta+\sin\theta'\cos\theta
                                   = -\sin(\theta-\theta')\\
    \vu{\theta'}\cdot\vu{r}       &= (-\sin\theta'\vu{x}+\cos\theta'\vu{y})\cdot(\cos\theta\vu{x}+\sin\theta\vu{y})
                                   = -\sin\theta'\cos\theta+\cos\theta'\sin\theta
                                   = \sin(\theta-\theta') \\
    \vu{\theta}'\cdot\vu{\theta}  &= (-\sin\theta'\vu{x}+\cos\theta'\vu{y})\cdot(-\sin\theta\vu{x}+\cos\theta\vu{y})
                                   = \sin\theta'\sin\theta+\cos\theta'\cos\theta
                                   = \cos(\theta-\theta')
  \end{align}
  \end{subequations}

To streamline the computation, we first compute the vector products contained
in each term of the integrands individually. This yields
  \begin{align}
    \vb{N}\times\vb{B}_\text{inc}
                &= \left[
                \left(n_\theta B_z-n_z B_\theta\right) \vu{r}
                    +\left(n_z B_r-n_r B_z\right)      \vu{\theta}
                    +\left(n_r B_\theta- n_\theta B_r\right)  \vu{z}
                  \right] \\
    \vb{N}\cdot\vb{E}_\text{inc}
                &= n_r E_r + n_\theta E_\theta + n_z E_z \\
    (\vb{N}\times\vb{B}_\text{inc})\times\vb{G}
                &= \left\{
                      \left[
                        \left( n_z B_r - n_r B_z \right) G_z
                        -\left( n_r B_\theta - n_\theta B_r \right) G_\theta
                      \right] \vu{r} \right. \nonumber\\
                &\quad+\left[
                        \left( n_r B_\theta - n_\theta B_r \right) G_r
                       -\left( n_\theta B_z - n_z B_\theta\right) G_z
                      \right]  \vu{\theta} \nonumber\\
                &\quad+\left.\left[
                    \left( n_\theta B_z-n_z B_\theta \right) G_\theta
                    -\left( n_z B_r - n_r B_z \right) G_r
                      \right]\vu{z}\right\}
  \end{align}
We can now extract the explicit expressions for the components of the reflected
electromagnetic fields, paying attention to the products of primed and unprimed
basis vectors:
%\todo[inline]{Check for multiline automatic sizing for \verb|\left| and \verb|\right|.}
  \begin{subequations}
  \begin{align}
    E_r'(\vb{r}', k)
          &= \vu{r}'\cdot\vb{E}'(\vb{r}',k) \nonumber\\
          &= 2\iint_S\left\{ikg \left[
              \left( n_\theta B_z -n_z B_\theta \right)\cos(\theta-\theta')
              +\left(n_r B_z -n_zB_r \right)         \sin(\theta-\theta')\right]\right.\nonumber\\
            &\quad \left.+\left(\frac{ik}{u}-\frac{1}{u^2}\right)g
                   \left(n_r E_r + n_\theta E_\theta + n_z E_z\right)\left(r\cos(\theta-\theta')-r'\right)\right\}dS \nonumber\\
            &\quad + \frac{2}{ik}\oint_{\partial S} \left(\frac{ik}{u}-\frac{1}{u^2}\right) g
                    \left(B_r l_r + B_\theta l_\theta + B_z l_z\right)\left(r\cos(\theta-\theta')-r'\right)dl,\\
   E_\theta'(\vb{r}',k)
          &=\vu{\theta}'\cdot\vb{E}'(\vb{r}',k) \nonumber\\
          &= 2\iint_S\left\{ikg \left[
              \left(n_\theta B_z - n_z B_\theta\right)\sin(\theta-\theta')
              +\left(n_z B_r  - n_r B_z\right)        \cos(\theta-\theta')\right]\right.\nonumber\\
            &\quad+\left(\frac{ik}{u}-\frac{1}{u^2}\right)g\left.\left(n_r E_r + n_\theta E_\theta + n_z E_z\right)
             r\sin(\theta-\theta')g\right\}dS \nonumber\\
            &\quad +\frac{2}{ik}\oint_{\partial S} \left(\frac{ik}{u}-\frac{1}{u^2}\right)g
                    \left(B_r l_r + B_\theta l_\theta + B_z l_z\right)r\sin(\theta-\theta')dl \\
  E_z'(\vb{r}',k)
          &= \vu{z}'\cdot\vb{E}'(\vb{r}',k) \nonumber\\
          &= 2\iint_S\left\{ikg \left(n_r B_\theta - n_\theta B_r\right)+\left(\frac{ik}{u}-\frac{1}{u^2}\right)\left( n_r E_r + n_\theta E_\theta + n_z E_z\right)(z-z')g\right\} dS\nonumber\\
            &\quad +\frac{2}{ik}\oint_{\partial S} \left(\frac{ik}{u}-\frac{1}{u^2}\right)g
              \left(B_r l_r + B_\theta l_\theta + B_z l_z\right)(z-z')dl
  \end{align}
The magnetic field gives
  \begin{align}
    B_r'(\vb{r}', k)
          &= \vu{r}'\cdot\vb{B}'(\vb{r}',k) \nonumber\\
          &= 2\iint_S \left(\frac{ik}{u}-\frac{1}{u^2}\right)g \left[
                \left(n_z B_r - n_r B_z \right) (z-z')\cos(\theta-\theta')\right.\nonumber\\
          &\quad\left.\left(
               +r\left(n_\theta B_z - n_z B_\theta\right)
               +\left(n_\theta B_z - n_z B_\theta\right)(z-z')\right)\sin(\theta-\theta')
               \right] \\
    B_\theta'(\vb{r}',k)
          &=\vu{\theta}'\cdot\vb{B}'(\vb{r}',k) \nonumber\\
          &= 2\iint_S \left(\frac{ik}{u}-\frac{1}{u^2}\right)g \left[
              \left(n_z B_r-n_r B_z\right)(z-z')\sin(\theta-\theta')\right. \nonumber\\
          &\quad +\left(n_z B_\theta - n_\theta B_z\right)(z-z')\cos(\theta-\theta') \nonumber\\
          &\quad\left.+\left(n_\theta B_r- n_r B_\theta\right)\left(r'-r\cos(\theta-\theta')\right)\right] dA \\
    B_z'(\vb{r}',k)
          &= \vu{z}'\cdot\vb{B}'(\vb{r}',k) \nonumber\\
          &= 2\iint_S \left(\frac{ik}{u}-\frac{1}{u^2}\right)g\left[
            \left(n_\theta B_z- n_z B_\theta\right)r'\sin(\theta-\theta') \right.\nonumber\\
          &\quad\left.+\left(n_r B_z - n_z B_r\right)\left(r-r'\cos(\theta-\theta')\right)\right] dA
  \end{align}
  \end{subequations}

For the given parametrization, the normal is given
  \begin{subequations}
  \begin{align}
    \vu{n} &= \frac{\nabla(z-F(r,\theta))}{\left|\nabla(z-F(r,\theta))\right|} \\
           &= \frac{\left[-\partial_r F(r,\theta)\vu{r}-\frac{1}{r}\partial_\theta F(r,\theta)\vu{\theta}+\vu{z}\right]}
                    {\sqrt{1+\left(\partial_r F(r,\theta\right)^2+\left(\frac{1}{r}\partial_\theta F(r,\theta)\right)^2}}
  \end{align}
while the surface element is given by
  \begin{equation}
    dS = dA\sqrt{1+\left(\partial_r F(r,\theta)\right)^2+\left(\frac{1}{r}\partial_\theta F(r,\theta)\right)^2}.
  \end{equation}
  \end{subequations}
Given that the normal vector is present only once in most of terms of the integrals,
the Jacobian due to the surface will conveniently cancel.

\chapter{Explicit Expressions for Four-Wave Mixing from the Euler-Heisenberg Lagrangian}

\todo[inline]{Appendix summary and references to main text.}

\section{}
It is possible to derive the polarization and magnetization vectors ($\vb{P}$ and $\vb{M}$)
that arise from the spontaneous creation and annihilation of virtual electron-positron
pairs from the Euler-Heisenberg (EH) Lagrangian. To do so, we use the weak-field
approximation of the EHL and compute the classical equations of motion that arise
from it. We can write
  \begin{subequations}
  \begin{align}
    \vb{P} &= \frac{4\alpha^2}{45}\left[ 4\mathcal{F}\vb{E}+7\mathcal{G}\vb{B}\right], \\
    \vb{M} &= \frac{4\alpha^2}{45}\left[-4\mathcal{F}\vb{B}+7\mathcal{G{\vb{E}}}\right]
  \end{align}
  \end{subequations}
where $\mathcal{F}$ and $\mathcal{G}$ are the Lorentz invariants defined in \label{somewhere}.

We can now derive equations of motions for the electromagnetic field from the
macroscopic version of Maxwell's equations:
  \begin{subequations}
  \begin{align}
    \nabla\times\vb{E} &= -\frac{\partial\vb{B}}{\partial t}, \\
    \nabla\times\vb{H} &=  \frac{\partial\vb{D}}{\partial t}, \\
    \nabla\cdot\vb{D}  &= 0, \\
    \nabla\cdot\vb{B}  &= 0
  \end{align}
  \end{subequations}
where the displacement field $\vb{D}$ and the magnetizing field $\vb{H}$ are
related to the polarization and magnetization $\vb{P}$ and $\vb{M}$ via
  \begin{subequations}
  \begin{align}
    \vb{D}  &=\vb{E}+\vb{P}, \\
    \vb{H}  &=\vb{B}-\vb{M}.
  \end{align}
  \end{subequations}
Taking the curl of the curl equations above and using the fact that the
divergence of $\vb{D}$ and $\vb{B}$ vanish, we can derive inhomogeneous
wave equations for both the electric and magnetic fields
  \begin{subequations}
  \begin{align}
    \left[\partial_t^2-\nabla^2\right]\vb{E}
                  &= -\partial_t\left[\nabla\times\vb{M}\right]-\partial_t^2\vb{P}+\nabla\left[\nabla\cdot\vb{P}\right], \\
    \left[\partial_t^2-\nabla^2\right]\vb{B}
                  &= \partial_t\left[\nabla\times\vb{P}\right]-\nabla^2\vb{M}+\nabla\left[\nabla\cdot\vb{M}\right].
  \end{align}
  \end{subequations}
As they stand, these equations are nonlinear partial differential equations and
can be incredibly difficult to solve. To simplify the process, we linearize these
equations by writing the total field as
  \begin{align}
    \vb{E}  &= \vb{E}_0 + \vb{E}_\text{gen} \\
    \vb{B}  &= \vb{B}_0 + \vb{B}_\text{gen}
  \end{align}
where $\vb{E}_0$ and $\vb{B}_0$ are solutions of the homogeneous wave equations
and $\vb{E}_\text{gen}$ and $\vb{B}_\text{gen}$ are the fields generated by the
non-linearity. We assume that $|\vb{E}_\text{gen}|\ll|\vb{E}_0|$ such that we
can write
  \begin{subequations}
  \begin{align}
    \left[\partial_t^2-\nabla^2\right]\vb{E}_\text{gen}
                  &= -\partial_t\left[\nabla\times\vb{M}_0\right]-\partial_t^2\vb{P}_0+\nabla\left[\nabla\cdot\vb{P}_0\right]
                  = \vb{F}_E \\
    \left[\partial_t^2-\nabla^2\right]\vb{B}_\text{gen}
                  &= \partial_t\left[\nabla\times\vb{P}_0\right]-\nabla^2\vb{M}_0+\nabla\left[\nabla\cdot\vb{M}_0\right]=\vb{F}_B
  \end{align}
  \end{subequations}
which is an inhomogeneous wave equation for $\left\{\vb{E}_0,\vb{B}_0\right\}$
with a \textit{known} source term. For our purposes, it will be useful to work
in frequency space. We write
  \begin{equation}
    \mathcal{F}\left[\vb{C}(\vb{r},t)\right] = \int_{-\infty}^\infty \vb{C}(\vb{r},t)e^{-i\omega t}d\omega
  \end{equation}
where $\vb{C}$ stands each for the fields in the PDE. Now, note that the polarization
and magnetization $\vb{P}_0$ and $\vb{M}_0$ are cubic functions of the applied fields
$\vb{E}_0$ and $\vb{B}_0$. As such, their Fourier expansion will involve taking
the cube of the Fourier expansion of the applied fields. We will explicitly evaluate
this operation a little later in this section.

Fourier transforming both sides of the PDEs and commuting the temporal derivatives
inside the integral sign yields, paying close attention to the signs
  \begin{subequations}
  \begin{align}
    \left[\nabla^2+k^2\right]\vb{E}_\text{gen}(\vb{r},\omega)
      &= i\omega\nabla\times\vb{M}_0(\vb{r},\omega) + \omega^2 \vb{P}_0(\vb{r},\omega)+\nabla\left[\nabla\cdot\vb{P}_0(\vb{r},\omega)\right], \\
    \left[\nabla^2+k^2\right]\vb{B}_\text{gen}(\vb{r},\omega)
      &= i\omega\nabla\times\vb{P}_0(\vb{r},\omega) + \nabla^2\vb{M}_0(\vb{r},\omega)-\nabla\left[\nabla\cdot\vb{M}_0(\vb{r},\omega)\right].
  \end{align}
  \end{subequations}
This yields inhomogeneous Helmholtz equations for the generated electric and
magnetic fields. They can be solved by using the Green's function of the
Helmholtz equation. Generically, we have
  \begin{equation}
    \vb{C}(\vb{r}',\omega)
      = \int_{\mathbb{R}^3} \vb{F}(\vb{r},\omega)\frac{e^{i\omega\left|\vb{r}-\vb{r}'\right|}}{4\pi\left|\vb{r}-\vb{r}'\right|}d^3\vb{r}.
  \end{equation}
Applying this solution to the generated electric and magnetic fields, we
can notice that this gives expressions with multiple spatial derivatives
to compute. These are difficult to evaluate numerically, but we can use the
far-field approximation and the chain rule to considerably simplify these
expressions.

For the far-field approximation to make sense, we will make the assumption
that the support $D$ of both $\vb{P}_0$ and $\vb{M}_0$ is smaller than a sphere of
radius $\ell$. This is not quite true, as the applied field falls off as
$1/R$, consequently the polarization and magnetization fall off as $1/R^3$.
In practice, we will vary the region of integration $D$ until some some predefined
convergence measure is achieved. In tightly focused fields configuration, $D$ will be
on the order of $\lambda^3$. The far-field approximation can thus be formulated
as the observation point $\vb{r}'$ being very far away from the applied field,
i.e. $|\vb{r}'|\gg\ell$. In the integrands, this will translate to $|\vb{r}|'\gg|\vb{r}|$.
We can expand the distance function using the law of cosines
  \begin{equation}
    |\vb{r}-\vb{r}'| = \sqrt{|\vb{r}|^2+|\vb{r}'|^2-2|\vb{r}||\vb{r}'|\cos\theta}\simeq |\vb{r}'|-\frac{1}{R}\vb{r}'\cdot\vb{r}.
  \end{equation}
Substituting this result in the Green's function yields
  \begin{equation}
    g(\vb{r},\vb{r}')\simeq \frac{e^{i\omega R}e^{-i\omega \vb{r}'\cdot\vb{r}/R}}{R}.
  \end{equation}

To remove the spatial derivatives, we use the following vector product
rules
Using this result in conjunction with product rules for vector calculus and volume-surface
integrals, we can write
  \begin{subequations}
  \begin{align}
    \psi\nabla\phi                &= -\nabla\psi\phi + \nabla(\psi\phi) \\
    \psi(\nabla\times\vb{C})      &= -\nabla\phi\times\vb{C} + \nabla\times(\psi\vb{C}) \\
    \psi(\nabla\cdot\vb{C})       &= -\vb{C}\cdot\nabla\psi + \nabla\cdot(\psi\vb{C}) \\
    \nabla\times\nabla\times\vb{C}&= \nabla[\nabla\cdot\vb{C}] - \nabla^2\vb{C}
  \end{align}
  \end{subequations}
and notice that since the integration is over the finite support of the integrands,
the surface terms that are generated by the total derivative terms vanish.
For the terms that contain second derivatives, notice that the gradient of the
Green's function does not actually depend on $\vb{r}$:
\begin{equation}
    \nabla_r g(\vb{r},\vb{r}') = -\frac{i\omega g}{R}\vb{r}'.
  \end{equation}
It is thus possible to simply reuse the vector product rule on the
remaining derivatives.
Finally, we can write
  \begin{subequations}
  \begin{align}
    \vb{E}_\text{gen}(\vb{r}',\omega)
        &= \frac{\omega^2e^{i\omega R}}{4\pi R}\int_D \left[-\vu{r}'\times \vb{M}_0(\vb{r},\omega)+\vb{P}_0(\vb{r},\omega)-\vu{r}'\left[\vu{r}'\cdot\vb{P}_0(\vb{r},\omega)\right]\right]e^{-i\omega \frac{\vb{r}'\cdot\vb{r}}{R}}d^3\vb{r}, \\
    \vb{B}_\text{gen}(\vb{r}',\omega)
        &= \frac{\omega^2e^{i\omega R}}{4\pi R}\int_D \left[+\vu{r}'\times \vb{P}_0(\vb{r},\omega)+\vb{M}_0(\vb{r},\omega)-\vu{r}'\left[\vu{r}'\cdot\vb{M}_0(\vb{r},\omega)\right]\right]e^{-i\omega \frac{\vb{r}'\cdot\vb{r}}{R}}d^3\vb{r},
  \end{align}
  \end{subequations}
which give the generated field at a far-field position $\vb{r}'$ as an integral
over the polarization and magnetization generated by the applied fields $\vb{E}_0$
and $\vb{B}_0$. Notice that, in this form, the expression for the magnetic field
can be obtain from the expression of the electric field by the mapping
$\vb{P}_0\mapsto\vb{M}_0$ and $\vb{M}_0\mapsto-\vb{P}_0$.

\section{Explicit expressions in cylindrical coordinates}

We now write down the explicit expressions of each component in cylindrical
coordinates. We will have to evaluate the cross-product of cylindrical
basis vectors evaluated at different points, similar to what we have done in
Appendix~\ref{}. We first write
  \begin{subequations}
  \begin{align}
    \vu{r}'\times\vu{r}           &= (\cos\theta'\vu{x}+\sin\theta'\vu{y})\times(\cos\theta\vu{x}+\sin\theta\vu{y})
                                   = (\cos\theta'\sin\theta-\sin\theta'\cos\theta)\vu{z}
                                   = \sin(\theta-\theta')\vu{z} \\
    \vu{r}'\times\vu{\theta}      &= (\cos\theta'\vu{x}+\sin\theta'\vu{y})\times(-\sin\theta\vu{x}+\cos\theta\vu{y})
                                   = (\cos\theta'\cos\theta+\sin\theta'\sin\theta)\vu{z}
                                   = \cos(\theta-\theta')\vu{z}\\
    \vu{r}'\times\vu{z}           &= -\vu{\theta}' \\
    \vu{\theta'}\times\vu{r}       &= (-\sin\theta'\vu{x}+\cos\theta'\vu{y})\times(\cos\theta\vu{x}+\sin\theta\vu{y})
                                   = (-\sin\theta'\sin\theta-\cos\theta'\cos\theta)\vu{z}
                                   = -\cos(\theta-\theta')\vu{z} \\
    \vu{\theta}'\times\vu{\theta} &= (-\sin\theta'\vu{x}+\cos\theta'\vu{y})\times(-\sin\theta\vu{x}+\cos\theta\vu{y})
                                   = (\sin\theta'\cos\theta+\cos\theta'\sin\theta)\vu{z}
                                   = \sin(\theta-\theta')\vu{z}\\
    \vu{\theta}'\times\vu{z}      &= \vu{r}' \\
    \vu{z}'\times\vu{r}           &= \vu{\theta} \\
    \vu{z}'\times\vu{\theta}      &= -\vu{r} \\
    \vu{z}'\times\vu{z}           &= 0.
  \end{align}
  \end{subequations}
Writing each term invidiually,
  \begin{subequations}
  \begin{align}
    \vb{r}'\times\vb{M}_0 &= \left(R_r\vu{r}'+R_z\vu{z}'\right)\times\left(M_r\vu{r}+M_\theta\vu{\theta}+M_z\vu{z}\right) \\
                          &= R_r\left(M_r\sin(\theta-\theta')+M_\theta\cos(\theta-\theta')\right)\vu{z}
                            -R_rM_z\vu{\theta}' + R_zMr\vu{\theta} -R_zM_\theta \vu{r} \\
    \vb{r}'\left(\vb{r}'\cdot\vb{P}\right) &=\left(R_r \vu{r}' + R_z \vu{z}'\right)
                                        \left(R_rP_r\cos(\theta-\theta')-R_rP_\theta\sin(\theta-\theta')+R_zP_z\right).
  \end{align}
  \end{subequations}
Evaluating the dot products yield
  \begin{subequations}
  \begin{align}
    E_{r,\text{gen}}(\vb{r}',\omega)
        &= \vu{r}'\cdot\vb{E}_\text{gen}(\vb{r}',\omega) \nonumber\\
        &= \frac{\omega^2e^{i\omega R}}{4\pi R} \int_D
        \left[
          R_zM_\theta\cos(\theta-\theta')+R_zM_r\sin(\theta-\theta')
          +P_r\cos(\theta-\theta')-P_\theta\sin(\theta-\theta')\right.\nonumber\\
        &  \phantom{\frac{\omega^2e^{i\omega R}}{4\pi R} \int_D}\left.-R_r \left(R_rP_r\cos(\theta-\theta')-R_rP_\theta\sin(\theta-\theta')+R_zP_z\right)\right]\exp\left[{-i\omega \frac{rr'\cos(\theta-\theta')+zz'}{R}}\right]d^3\vb{r} \\
   E_{\theta,\text{gen}}(\vb{r}',\omega)
        &= \vu{\theta}'\cdot\vb{E}_\text{gen}(\vb{r}',\omega) \nonumber\\
        &= \frac{\omega^2e^{i\omega R}}{4\pi R} \int_D
        \left[
          R_rM_z-R_zM_r\cos(\theta-\theta')+R_zM_\theta\sin(\theta-\theta')+P_r\sin(\theta-\theta')+P_\theta\cos(\theta-\theta')\right]\exp\left[{-i\omega \frac{rr'\cos(\theta-\theta')+zz'}{R}}\right]d^3\vb{r}\\
  E_{z,\text{gen}}(\vb{r}',\omega)
        &= \vu{z}'\cdot\vb{E}_\text{gen}(\vb{r}',\omega) \nonumber\\
        &= \frac{\omega^2e^{i\omega R}}{4\pi R} \int_D
          \left[-R_rM_r\sin(\theta-\theta')-R_rM_\theta\cos(\theta-\theta')+P_z \right.\nonumber\\
        &\phantom{\frac{\omega^2e^{i\omega R}}{4\pi R} \int_D}\left.
          -R_z \left(R_rP_r\cos(\theta-\theta')-R_rP_\theta\sin(\theta-\theta')+R_zP_z\right)\right]
          \exp\left[{-i\omega \frac{rr'\cos(\theta-\theta')+zz'}{R}}\right]d^3\vb{r}.
  \end{align}
As mentioned earlier, the magnetic field can readily be obtained by the simple
mapping $\vb{P}_0\mapsto\vb{M}_0$ and $\vb{M}_0\mapsto-\vb{P}_0$.
  \begin{align}
    B_{r,\text{gen}}(\vb{r}',\omega)
        &= \vu{r}'\cdot\vb{B}_\text{gen}(\vb{r}',\omega) \nonumber\\
        &= \frac{\omega^2e^{i\omega R}}{4\pi R} \int_D
        \left[
          -R_zM_\theta\cos(\theta-\theta')-R_zP_r\sin(\theta-\theta')
          +M_r\cos(\theta-\theta')-M_\theta\sin(\theta-\theta')\right.\nonumber\\
        &  \phantom{\frac{\omega^2e^{i\omega R}}{4\pi R} \int_D}\left.-R_r \left(R_rM_r\cos(\theta-\theta')-R_rM_\theta\sin(\theta-\theta')+R_zM_z\right)\right]\exp\left[{-i\omega \frac{rr'\cos(\theta-\theta')+zz'}{R}}\right]d^3\vb{r} \\
   B_{\theta,\text{gen}}(\vb{r}',\omega)
        &= \vu{\theta}'\cdot\vb{B}_\text{gen}(\vb{r}',\omega) \nonumber\\
        &= \frac{\omega^2e^{i\omega R}}{4\pi R} \int_D
        \left[
          -R_rP_z+R_zP_r\cos(\theta-\theta')-R_zP_\theta\sin(\theta-\theta')+M_r\sin(\theta-\theta')+M_\theta\cos(\theta-\theta')\right]\exp\left[{-i\omega \frac{rr'\cos(\theta-\theta')+zz'}{R}}\right]d^3\vb{r}\\
   B_{z,\text{gen}}(\vb{r}',\omega)
        &= \vu{z}'\cdot\vb{B}_\text{gen}(\vb{r}',\omega) \nonumber\\
        &= \frac{\omega^2e^{i\omega R}}{4\pi R} \int_D
          \left[R_rP_r\sin(\theta-\theta')+R_rP_\theta\cos(\theta-\theta')+M_z \right.\nonumber\\
        &\phantom{\frac{\omega^2e^{i\omega R}}{4\pi R} \int_D}\left.
          -R_z \left(R_rM_r\cos(\theta-\theta')-R_rM_\theta\sin(\theta-\theta')+R_zM_z\right)\right]
          \exp\left[{-i\omega \frac{rr'\cos(\theta-\theta')+zz'}{R}}\right]d^3\vb{r}.
  \end{align}
  \end{subequations}
\todo[inline]{Import command that pushes equations to the end of the line.}

\section{Modes}


%\chapter{Other stuff to put in appendices}

\end{document}