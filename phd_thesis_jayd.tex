% --------------------------------------------------------------------------- %
% Author:          Joey Dumont                <joey.dumont@gmail.com>         %
% Date created:    Mar. 8th, 2017                                             %
% Description:     Ph. D. thesis file.                                        %
% License:         CC0                                                        %
%                  <https://creativecommons.org/publicdomain/zero/1.0>        %
% --------------------------------------------------------------------------- %

% --------------------------------------------------------------------------- %
% --                               Preamble                                -- %
% --------------------------------------------------------------------------- %

% ----------------------------------------------------------------- %
% --                       Document Class                        -- %
% ----------------------------------------------------------------- %

\documentclass[11pt,SymmetricalJury]{inrsthesis}

% ----------------------------------------------------------------- %
% --                          Packages                           -- %
% ----------------------------------------------------------------- %

\usepackage[unicode=true,
      pdfauthor={Joey Dumont},
      pdftitle={},
      bookmarks=true,
      bookmarksnumbered=true,
      bookmarksopen=true,
      bookmarksopenlevel=3,
      breaklinks=false,
      pdfborder={0 0 0},
      backref=false,
      colorlinks=true,
      linktoc=page,
      linkcolor=red,
      citecolor=blue,
      urlcolor=blue]
 {hyperref}


\title{Strong-field quantum electrodynamics \\ in tightly focused fields}
%\subtitle{Towards a Realitic Modelling of High-Power \\Laser Systems in the Quantum Theory}
\author{Joey Dumont}
\year{2018}
\program{Sciences de l'énergie et des matériaux}
\centreINRS{Centre Énergie Matériaux et Télécommunications}
\jury{
  \juryitem
    {Président du jury et \\ examinateur interne}
    {Nom du professeur \\ Institution}
  \\
  \juryitem
    {Examinateur externe}
    {Nom du professeur \\ Faculté ou département \\ Institution}
  \\
  \juryitem
    {Examinateur interne}
    {Nom du professeur \\ Institution}
  \\
  \juryitem
    {Directeur de recherche}
    {Jean-Claude Kieffer \\ INRS-ÉMT}
  \\
  \juryitem
    {Codirecteur de recherche}
    {Steve MacLean \\ INRS-ÉMT \\ University of Waterloo}
}
% --------------------------------------------------------------------------- %
% --                               Document                                -- %
% --------------------------------------------------------------------------- %

\begin{document}

\frontmatter

\maketitle

\chapter{Résumé}

Mots-clés:

\chapter{Abstract}

Keyword:

\chapter{Sommaire récapitulatif}
\cleardoublepage

\tableofcontents
\cleardoublepage

\listoftables
\cleardoublepage

\listoffigures
\cleardoublepage

% \dedication{To my wife and son.}
% \cleardoublepage

% \epigraph{}{}
% \cleardoublepage

\mainmatter

\chapter{Introduction}

Basic plan:
  \begin{itemize}
    \item Generic introduction to SF-QED, experiments by SLAC and the upcoming
          high power-laser facilities.
    \item Motivation to accurately model the tight focusing regime: field inhomogeneities
          play an important role in the detection of SF-QED observables (cite recent work by Di Piazza).
    \item Approach based on two steps: modelling the optical side of things, then the quantum side of things.
    \item Motivation behind the StrattoCalculator (vs Richards-Wolf, vs analytical solutions).
    \item Motivation behind SK (vs effective methods, vs analytical solutions). BONUS: inclusive observables.
    \item Discussion of the use of HPC resources as a driver of future SF-QED experiments?
  \end{itemize}

Stuff to put in appendices:
  \begin{enumerate}
    \item Details of the Stratton-Chu formalism, perhaps some stuff on
          discontinuity ``problem'' \cite{Asvestas1980} and the heuristic argument as to why
          the physical approximation is correct.
    \item Expressions of identity operators in Fock space for photons and fermions.
    \item Proof that disconnected vacuum diagrams vanish identically in the SK formalism.
    \item Minor results related to the \texttt{StrattoCalculator}:
      \begin{itemize}
        \item Gouy phase and its transverse momentum interpretation.
        \item Work on the generation of pseudo-radially polarized beams with mosaics of half-wave plates?
      \end{itemize}
  \end{enumerate}

\chapter{Tool 1: The Stratton-Chu Diffraction Integrals}

\chapter{Application I: Four-Wave Mixing \textit{in vacuo} and Pair Production}

\chapter{Application II: Radiation Reaction With High Energy Electrons}

\chapter{Tool 2: The Schwinger-Keldysh Formalism and SF-QED Observables}

\chapter{Application III: Photon Production via Compton Scattering}

\chapter{Application IV: Pair Production via Compton Scattering}

\chapter{Conclusion}

%\backmatter

\bibliographystyle{osajnl}
\bibliography{phd_thesis}

\appendix

\chapter{Conventions}

\chapter{Other stuff to put in appendices}

\end{document}
