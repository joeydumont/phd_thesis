% --------------------------------------------------------------------------- %
% Author:          Joey Dumont                <joey.dumont@gmail.com>         %
% Date created:    Mar. 8th, 2017                                             %
% Description:     Ph. D. thesis file.                                        %
% License:         CC0                                                        %
%                  <https://creativecommons.org/publicdomain/zero/1.0>        %
% --------------------------------------------------------------------------- %

% --------------------------------------------------------------------------- %
% --                               Preamble                                -- %
% --------------------------------------------------------------------------- %

% ----------------------------------------------------------------- %
% --                       Document Class                        -- %
% ----------------------------------------------------------------- %

\documentclass[11pt,SymmetricalJury]{inrsthesis}

% ----------------------------------------------------------------- %
% --                          Packages                           -- %
% ----------------------------------------------------------------- %

\usepackage[unicode=true,
      pdfauthor={Joey Dumont},
      pdftitle={},
      bookmarks=true,
      bookmarksnumbered=true,
      bookmarksopen=true,
      bookmarksopenlevel=3,
      breaklinks=false,
      pdfborder={0 0 0},
      backref=false,
      colorlinks=true,
      linktoc=page,
      linkcolor=red,
      citecolor=blue,
      urlcolor=blue]
 {hyperref}

% ----------------------------------------------------------------- %
% --                        Customization                        -- %
% ----------------------------------------------------------------- %
\newenvironment{chaptersummary}{%
  \begin{quotation}
  \SingleSpacing
  \setlength{\parskip}{\baselineskip}}{%
  \end{quotation}}

% ----------------------------------------------------------------- %
% --                       Title Page Info                       -- %
% ----------------------------------------------------------------- %
\title{Strong-field quantum electrodynamics \\ in tightly focused fields}
%\subtitle{Towards a Realitic Modelling of High-Power \\Laser Systems in the Quantum Theory}
\author{Joey Dumont}
\year{2018}
\program{Sciences de l'énergie et des matériaux}
\centreINRS{Centre Énergie Matériaux et Télécommunications}
\jury{
  \juryitem
    {Président du jury et \\ examinateur interne}
    {Nom du professeur \\ Institution}
  \\
  \juryitem
    {Examinateur externe}
    {Nom du professeur \\ Faculté ou département \\ Institution}
  \\
  \juryitem
    {Examinateur interne}
    {Nom du professeur \\ Institution}
  \\
  \juryitem
    {Directeur de recherche}
    {Jean-Claude Kieffer \\ INRS-ÉMT}
  \\
  \juryitem
    {Codirecteur de recherche}
    {Steve MacLean \\ INRS-ÉMT \\ University of Waterloo}
}
% --------------------------------------------------------------------------- %
% --                               Document                                -- %
% --------------------------------------------------------------------------- %

\begin{document}

\frontmatter

\maketitle

\chapter{Résumé}

Mots-clés:

\chapter{Abstract}

Keyword:

\chapter{Sommaire récapitulatif}
\cleardoublepage

\tableofcontents
\cleardoublepage

\listoftables
\cleardoublepage

\listoffigures
\cleardoublepage

% \dedication{To my wife and son.}
% \cleardoublepage

\epigraph{More important, the formulation in terms of a variational principle is the route
          that is generally followed when we try to describe apparently nonmechanical systems
          in the mathematical clothes of classical mechanics, as in the theory of fields.}{Goldstein \textit{et al}, Classical Mechanics, 3rd edition.}
\cleardoublepage

\mainmatter

\chapter{Introduction}

Basic plan:
  \begin{itemize}
    \item Generic introduction to SF-QED, experiments by SLAC and the upcoming
          high power-laser facilities.
    \item Motivation to accurately model the tight focusing regime: field inhomogeneities
          play an important role in the detection of SF-QED observables (cite recent work by Di Piazza).
    \item Approach based on two steps: modelling the optical side of things, then the quantum side of things.
    \item Motivation behind the StrattoCalculator (vs Richards-Wolf, vs analytical solutions).
    \item Motivation behind SK (vs effective methods, vs analytical solutions). BONUS: inclusive observables.
    \item Discussion of the use of HPC resources as a driver of future SF-QED experiments?
  \end{itemize}

Stuff to put in appendices:
  \begin{enumerate}
    \item Details of the Stratton-Chu formalism, perhaps some stuff on
          discontinuity ``problem'' \cite{Asvestas1980} and the heuristic argument as to why
          the physical approximation is correct.
    \item Expressions of identity operators in Fock space for photons and fermions.
    \item Proof that disconnected vacuum diagrams vanish identically in the SK formalism.
    \item Minor results related to the \texttt{StrattoCalculator}:
      \begin{itemize}
        \item Gouy phase and its transverse momentum interpretation.
        \item Work on the generation of pseudo-radially polarized beams with mosaics of half-wave plates?
      \end{itemize}
  \end{enumerate}

\chapter{Tool 1: The Stratton-Chu Diffraction Integrals}

\begin{chaptersummary}
This chapter discusses the problem of properly modelling electromagnetic fields
in the tight focusing regime. We first lay the foundation of electromagnetic
scattering and justify the choice of the Stratton-Chu formalism. Our numerical
implementation of this formalism is then discussed at length, and our results
are benchmarked against analytical results and results from the literature.
We briefly discuss the Richards-Wolf formalism, which is similar but not equivalent
to the Stratton-Chu formulation.
\end{chaptersummary}

Skeleton:
  \begin{itemize}
      \item Short derivation of the Stratton-Chu equations, with emphasis on the physical optics approximation.
      \item Numerical implementation of the method
        \begin{itemize}
            \item Pros/cons vs more popular methods.
            \item Details of the field calculation (field models, normalization and whatnot).
            \item Details of the MPI implementation.
            \item Phase cancellation in the focal spot.
            \item Appendix: Numerical checks like convergence, solution of Maxwell's equations.
            \item Appendix: Far-field of the SC equations.
            \item Appendix: Analysis of the Gouy phase and its transverse momentum interpretation.
        \end{itemize}
      \item Comparison with Richards-Wolf:
        \begin{itemize}
            \item No ``real'' derivation from SC to RW without implicit approximations.
            \item Problem with phase cancellation?
            \item Differing results in similar conditions.
        \end{itemize}
      \item Analysis of the fields in the focal spot and comparison with
            existing analytical solutions (Salamin, e-dipoles).
      \item Opening to other methods that are less computationnaly intensive;
        \begin{itemize}
            \item Transformation optics;
            \item Propagation equations?
        \end{itemize}
  \end{itemize}


\chapter{Application I: Four-Wave Mixing \textit{in vacuo} and Pair Production}

Skeleton:
\begin{itemize}
    \item Simple derivation of the EH Lagrangian.
    \item How to extract FWM from EH: linearisation + integral solution (WaveMixer)
    \item How to extract PP from EH: Adiabatic formula.
    \item MASTER TABLE, or how different field models affect FWM.
      \begin{itemize}
        \item Introduction of the different field models.
        \item Introduction of the transmission parabola.
        \item Explain the behaviour of yield as a function of 2f/rmax with Lorentz invariants,
              i.e. the optics of the experiment.
      \end{itemize}
    \item Pair production, w/ linear and radial beams.
      \begin{itemize}
          \item Show that both polarizations show the same behaviour as a function of f.
          \item Visualize the 3D fields and explain optimum in terms of the optics (cancellation of the
                magnetic field in both cases).
      \end{itemize}
\end{itemize}

\chapter{Application II: Radiation Reaction With High Energy Electrons}

Skeleton (still a rough idea):
  \begin{itemize}
    \item Quantify RR in different field configurations, checking whether there
          is an advantage to using a transmission parabola in this case.
    \item Study the effects of different field models on the detectability of
          RR effects.
    \item Mention intensity measurement stuff? I've worked a little on this\ldots
  \end{itemize}

\chapter{Tool 2: The Schwinger-Keldysh Formalism and SF-QED Observables}

% On the mathematical nonexistence of path integrals.
\epigraph{The emperor has no clothes.}{Yours truly}


Skeleton (early stages):
  \begin{itemize}
      \item Motivation of SK: single+double Compton scattering in Furry picture,
            difficulty of going above double/including the effects of arbitrary fields.
      \item Generalities of the SK formalism (definitions and formal results).
      \item Application to SF-QED: computing inclusive observables in
            arbitrary fields (proof-of-concept derivation following Gelis).
      \item Discuss the generality of the method: compatibility with
            Volkov solns, with the Di Piazza method, \textit{and} with numerical
            solutions of the Dirac equation.
      \item Appendix: Disconnected vacuum diagrams vanish in SK.
      \item Appendix: Proper derivation of the Volkov-SK propagators.
  \end{itemize}


\chapter{Application III: Photon Production via Compton Scattering (Trident/Breit-Wheeler)}

Work to be done:
  \begin{itemize}
      \item Formal derivation of the average number of photons with an electron
            in the initial state.
      \item Computation with Volkov propagators.
      \item Computation with Di Piazza solution.
      \item Opening to numerical solutions?
\end{itemize}

\chapter{Application IV: Pair Production via Compton Scattering}

Work to be done:
See previous chapter but with number of pairs through ``secondary'' processes.

\chapter{Conclusion}

%\backmatter

\bibliographystyle{osajnl}
\bibliography{phd_thesis}

\appendix

\chapter{Conventions or w/e}

%\chapter{Other stuff to put in appendices}

\end{document}
